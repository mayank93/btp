\chapter{Analysis Of Existing Algorithms}
\label{Chapter2}
\lhead{Chapter 2. \emph{Analysis Of Existing Algorithms}}

\section{CLIQUE, PROCLUS and ORCLUS}

One of the first algorithms in the area of projected clustering
algorithms is CLIQUE \cite{bib3}. CLIQUE finds the dense regions
(clusters) in a level-wise manner, based on the Apriori
principle. However, this algorithm does not scale well with
data dimensionality. In addition, the formed clusters have
large overlap, and this may not generate clear disjoint partitions. PROCLUS\cite{bib1} and ORCLUS\cite{bib2} employ alternative
techniques. They are much faster than CLIQUE and they
can discover disjoint clusters. In PROCLUS, the dimensions
relevant to each cluster are selected from the original set of
attributes. ORCLUS is more general and can select relevant attributes from the set of arbitrary directed orthogonal
vectors. PROCLUS fails to identify clusters with large difference in size and requires their dimensionality to be in a
predefined range. ORCLUS may discover clusters that are
hard to interpret.

\section{DOC}

DOC \cite{bib12} is a density-based algorithm that iteratively discovers projected clusters in a data set. DOC discovers one
cluster at a time. At each step, it tries to guess a good
medoid for the next cluster to be discovered. It repeatedly
picks a random point from the database and attempts to
discover the cluster centered at random point. Among all
discovered clusters, the cluster with the highest quality is selected. The process repeated for number of random points.

\section{Frequent Pattern-based Algorithm/MineClus}

The frequent pattern mining\cite{bib7} is one of the interesting
algorithms in data mining. The frequent itemset mining
problem was first proposed in\cite{bib4}. The frequent patterns
are generated from the transactional databases for a given
minsup. Subsequently, an efficient approach was proposed
in FP-growth\cite{bib8} to generate the frequent patterns without
generating the candidate items. In\cite{bib18, bib19}, a projected clustering algorithm was proposed. The algorithm iteratively
produce one cluster at a time by exploiting the frequent pattern mining approach. The top supported frequent patterns
are employed to find the item sets to generate the clusters.
A strength function is defined to arrange the clusters in to
strong and weak clusters. The strong clusters are produced
as an output and the objects in the weak clusters are added
by to the database. The process is iterated till all the objects
are grouped into the clusters. The FP-growth approach has
employed to generate the frequent patterns.
