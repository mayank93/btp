%% ----------------------------------------------------------------
%% Thesis.tex -- MAIN FILE (the one that you compile with LaTeX)
%% ---------------------------------------------------------------- 

% Set up the document
\documentclass[a4paper, 11pt, oneside]{Thesis}  % Use the "Thesis" style, based on the ECS Thesis style by Steve Gunn
\graphicspath{{Figures/}}  % Location of the graphics files (set up for graphics to be in PDF format)

% Include any extra LaTeX packages required
\usepackage[square, numbers, comma, sort&compress]{natbib}  % Use the "Natbib" style for the references in the Bibliography
\usepackage{verbatim}  % Needed for the "comment" environment to make LaTeX comments
\usepackage{vector}  % Allows "\bvec{}" and "\buvec{}" for "blackboard" style bold vectors in maths
\usepackage{textcomp} % Copyright symbol
\usepackage[]{algorithm2e}
\RestyleAlgo{algoruled}
\hypersetup{urlcolor=black, colorlinks=true}  % Colours hyperlinks in blue, but this can be distracting if there are many links.

%% ----------------------------------------------------------------
\begin{document}
\frontmatter	  % Begin Roman style (i, ii, iii, iv...) page numbering

% Set up the Title Page
\title  {Application of DiverseRank to Clustering}
%\authors  {\texorpdfstring
%{\href{your web site or email address}{Mayank Gupta}}
%{Mayank Gupta}
%}
\authors {Mayank Gupta\\201101004\\ \texttt{mayank.g@students.iiit.ac.in} \\ Ankush Jain\\201101010\\ \texttt{ankush.jain@students.iiit.ac.in}}
\addresses  {\groupname\\\deptname\\\univname}  % Do not change this here, instead these must be set in the "Thesis.cls" file, please look through it instead
\date       {\today}
\subject    {}
\keywords   {}

\maketitle
%% ----------------------------------------------------------------

\setstretch{1.3}  % It is better to have smaller font and larger line spacing than the other way round

% Define the page headers using the FancyHdr package and set up for one-sided printing
\fancyhead{}  % Clears all page headers and footers
\rhead{\thepage}  % Sets the right side header to show the page number
\lhead{}  % Clears the left side page header

\pagestyle{fancy}  % Finally, use the "fancy" page style to implement the FancyHdr headers

%% ----------------------------------------------------------------
% Copyright
\pagestyle{empty}  % No headers or footers for the following pages

\null\vfill
% Now comes the "Funny Quote", written in italics
%\textit{``Write a funny quote here.''}
\begin{center}
	Copyright {\textcopyright} Mayank Gupta, Ankush Jain, 2014 \\
	All Rights Reserved \\
\end{center}

%\begin{flushright}
%If the quote is taken from someone, their name goes here
%\end{flushright}

\vfill\vfill\vfill\vfill\vfill\vfill\null
\clearpage  % Funny Quote page ended, start a new page
%% ----------------------------------------------------------------
% Declaration Page required for the Thesis, your institution may give you a different text to place here
\Declaration{

	\addtocontents{toc}{\vspace{1em}}  % Add a gap in the Contents, for aesthetics

	\begin{center}
		It is certified thad the work contained in this project, titled \\
		\emph{"Application of DiverseRank to Classification"} \\ by \underline{Mayank Gupta} with Roll No. \underline{201101004} and \\
		\underline{Ankush Jain} with Roll No. \underline{201101010}\\
		has been carried out under my supervision for partial fulfillment of degree.\par
	\end{center}

	%I, AUTHOR NAME, declare that this thesis titled, `THESIS TITLE' and the work presented in it are my own. I confirm that:

	%\begin{itemize} 
	%\item[\tiny{$\blacksquare$}] This work was done wholly or mainly while in candidature for a research degree at this University.

	%\item[\tiny{$\blacksquare$}] Where any part of this thesis has previously been submitted for a degree or any other qualification at this University or any other institution, this has been clearly stated.

	%\item[\tiny{$\blacksquare$}] Where I have consulted the published work of others, this is always clearly attributed.

	%\item[\tiny{$\blacksquare$}] Where I have quoted from the work of others, the source is always given. With the exception of such quotations, this thesis is entirely my own work.

	%\item[\tiny{$\blacksquare$}] I have acknowledged all main sources of help.

	%\item[\tiny{$\blacksquare$}] Where the thesis is based on work done by myself jointly with others, I have made clear exactly what was done by others and what I have contributed myself.
	%\\
	%\end{itemize}


	Signed:\\
	\rule[1em]{25em}{0.5pt}  \\ % This prints a line for the signature
	(Adviser: Dr. P. Krishna Reddy)

	Date:\\
	\rule[1em]{25em}{0.5pt}  % This prints a line to write the date
}
\clearpage  % Declaration ended, now start a new page

%% ----------------------------------------------------------------
% The "Funny Quote Page"
\pagestyle{empty}  % No headers or footers for the following pages

\null\vfill
% Now comes the "Funny Quote", written in italics
\textit{``Winter is coming.''}

\begin{flushright}
	{\textemdash} Lord Eddard Stark
\end{flushright}

\vfill\vfill\vfill\vfill\vfill\vfill\null
\clearpage  % Funny Quote page ended, start a new page
%% ----------------------------------------------------------------

% The Abstract Page
\addtotoc{Abstract}  % Add the "Abstract" page entry to the Contents
\abstract{
	\addtocontents{toc}{\vspace{1em}}  % Add a gap in the Contents, for aesthetics

	The process of clustering divides a set of data objects into smaller sets consisting of similar objects.

	These clustering algorithms do not work efficiently in higher dimensional spaces because of the inherent sparsity of the data.

	Projected clustering algorithms have been proposed to find the clusters in hidden subspaces. The identification of more relevant attributes is a research issue in the projected clustering.

	In literature, an approach has been proposed to find the relevant subspaces using Frequent-Pattern Mining approaches.
	The frequent pattern can able to identify the item sets with high support.
	In frequent pattern based projected clustering approach, we observed that only frequent itemsets are considered.

	In this paper, we propose an approach to refine clusters generated by the frequent pattern based approach and produce more accurate clusters.
	In the proposed approach, we identify the irrelevant objects in a given cluster by exploiting the notion of diversity.
	Based on the notion of diversity, which captures the extent the items in the pattern belong to multiple categories of a concept hierarchy, the objects the membership of an object is identified.
	The membership of each object decides the existance of the object in the cluster.
	This process iteratively refines each cluster.
	We conduct the experiments on the real world datasets and show that the proposed approach improves the quality of the cluster

}

\clearpage  % Abstract ended, start a new page
%% ----------------------------------------------------------------

\setstretch{1.3}  % Reset the line-spacing to 1.3 for body text (if it has changed)

% The Acknowledgements page, for thanking everyone
\acknowledgements{
	\addtocontents{toc}{\vspace{1em}}  % Add a gap in the Contents, for aesthetics

	%The acknowledgements and the people to thank go here, don't forget to include your project advisor\ldots
	We are grateful to Mr. M. Kumaraswamy and Dr. P. Krishna Reddy for their constant help and guidance throughout the project.

}
\clearpage  % End of the Acknowledgements
%% ----------------------------------------------------------------

\pagestyle{fancy}  %The page style headers have been "empty" all this time, now use the "fancy" headers as defined before to bring them back


%% ----------------------------------------------------------------
\lhead{\emph{Contents}}  % Set the left side page header to "Contents"
\tableofcontents  % Write out the Table of Contents

%%% ----------------------------------------------------------------
%\lhead{\emph{List of Figures}}  % Set the left side page header to "List if Figures"
%\listoffigures  % Write out the List of Figures

%%% ----------------------------------------------------------------
%\lhead{\emph{List of Tables}}  % Set the left side page header to "List of Tables"
%\listoftables  % Write out the List of Tables

%%% ----------------------------------------------------------------
%\setstretch{1.5}  % Set the line spacing to 1.5, this makes the following tables easier to read
%\clearpage  % Start a new page
%\lhead{\emph{Abbreviations}}  % Set the left side page header to "Abbreviations"
%\listofsymbols{ll}  % Include a list of Abbreviations (a table of two columns)
%{
%% \textbf{Acronym} & \textbf{W}hat (it) \textbf{S}tands \textbf{F}or \\
%\textbf{LAH} & \textbf{L}ist \textbf{A}bbreviations \textbf{H}ere \\

%}

%%% ----------------------------------------------------------------
%\clearpage  % Start a new page
%\lhead{\emph{Physical Constants}}  % Set the left side page header to "Physical Constants"
%\listofconstants{lrcl}  % Include a list of Physical Constants (a four column table)
%{
%% Constant Name & Symbol & = & Constant Value (with units) \\
%Speed of Light & $c$ & $=$ & $2.997\ 924\ 58\times10^{8}\ \mbox{ms}^{-\mbox{s}}$ (exact)\\

%}

%%% ----------------------------------------------------------------
%\clearpage  %Start a new page
%\lhead{\emph{Symbols}}  % Set the left side page header to "Symbols"
%\listofnomenclature{lll}  % Include a list of Symbols (a three column table)
%{
%% symbol & name & unit \\
%$a$ & distance & m \\
%$P$ & power & W (Js$^{-1}$) \\
%& & \\ % Gap to separate the Roman symbols from the Greek
%$\omega$ & angular frequency & rads$^{-1}$ \\
%}
%%% ----------------------------------------------------------------
% End of the pre-able, contents and lists of things
% Begin the Dedication page

\setstretch{1.3}  % Return the line spacing back to 1.3

\pagestyle{empty}  % Page style needs to be empty for this page
\dedicatory{
	Dedicated to Shubham Sangal\ldots
}

\addtocontents{toc}{\vspace{2em}}  % Add a gap in the Contents, for aesthetics


%% ----------------------------------------------------------------
\mainmatter	  % Begin normal, numeric (1,2,3...) page numbering
\pagestyle{fancy}  % Return the page headers back to the "fancy" style

% Include the chapters of the thesis, as separate files
% Just uncomment the lines as you write the chapters

%\input{./Chapters/Chapter1} % Introduction

\chapter{Introduction}
\label{Chapter1}
\lhead{Chapter 1. \emph{Introduction}}

Clustering is a Data Mining technique, that, given a dataset, divides it into
groups of objects such that the objects in a group will be
similar to one another and different from the objects in other
groups\cite{bib17}. The objects are represented as a point, where
each dimension corresponds to an attribute/feature and the
feature value of each object determines its coefficient in the
corresponding dimension. The data points are grouped into
clusters based on the some similarity metric. The clustering
algorithms have large number of applications such as marketing (e.g., customer segmentation), image analysis, bio-
informatics, document classification, indexing, etc.

\section{Issues with Clustering Techniques}

Most clustering algorithms do not work efficiently in higher
dimensional space because of the inherent sparsity of the
data \cite{bib3}. In high dimensional space, it is likely that the
distance of any two points is almost the same for a large
class of common distributions \cite{bib5}. So, a clustering algorithm
employ feature selection based approach know as subspaces
clustering \cite{bib11}. The goal of the subspace clustering is to
find the particular dimensions on which the data points are
correlated and pruning away the remaining dimensions that
reduces the noise in the data.

The problem in these algorithms is that picking certain dimensions in advance can
lead to a loss of information. Furthermore, in many real
data, some points are correlated with respect to a given set
of dimensions and others are correlated with respect to different dimensions. Thus, it may not always be feasible to
prune off too many dimensions without considering the data
at the same time incurring a substantial loss of information.
Alternatively, the effects of dimensionality can be reduced
by a dimensionality reduction technique\cite{bib6}, however, in-
formation from all dimensions is uniformly transformed and
relevant information for some clusters may be reduced. Also,
the clusters may be hard to understand.

\section{Available Algorithms}

The projected clustering\cite{bib1} alogrithms overcome some of the
issues of the subspace clustering. In projected clustering, a
set of data points with an associated set of relevant dimensions are employed such that the data points are similar to
each other in the subspace formed by the relevant dimensions, but dissimilar to data point outside the cluster. The
widely used distance measures are more meaningful in projections of the high-dimensional space, where the object
values are dense\cite{bib9}. In other words, it is more likely for the data
to form dense, meaningful clusters in a high-dimensional
subspace. CLIQUE \cite{bib3}, PROCLUS \cite{bib1}, ORCLUS \cite{bib2}, and DOC \cite{bib12} are some efforts in the projected clustering approaches.

These approaches suffer from the quality of the clusters and
consumes lot of time. An effort is made to exploit the frequent pattern mining in the area of projected clustering to
address the issue of quality clusters\cite{bib18, bib19}. However, the
frequent pattern mining\cite{bib4} approach identifies the itemsets
(features/attributes) with high support. In frequent pattern
based projected clustering approach, we observed that only
frequent itemsets are considered and the infrequent itemsets
are completely eliminated from the clustering process.

\section{Our Approach}

We propose an approach to refine the clusters
generated by the frequent pattern based projected clustering approach and produce more qualitative clusters. We
observed that a high support pattern includes few items
and eliminates non-frequent items.
As a result of elimination of some of the items, the influence of other items
is not considered in the process of clustering. Due to elimination of some
data values, the the quality of clusters gets affected. In the
proposed approach, for a given cluster, we identify the mem-
bership of each data point employing the notion called diversity. The diversity identify the non-members of the given
cluster and remove the objects from the cluster. We conduct
the experiment on the real world datasets and show that the
proposed approach improves the quality of clusters.
 % Background Theory 

\chapter{Analysis Of Existing Algorithms}
\label{Chapter2}
\lhead{Chapter 2. \emph{Analysis Of Existing Algorithms}}

\section{CLIQUE, PROCLUS and ORCLUS}

One of the first algorithms in the area of projected clustering
algorithms is CLIQUE \cite{bib3}. CLIQUE finds the dense regions
(clusters) in a level-wise manner, based on the Apriori
principle. However, this algorithm does not scale well with
data dimensionality. In addition, the formed clusters have
large overlap, and this may not generate clear disjoint partitions. PROCLUS\cite{bib1} and ORCLUS\cite{bib2} employ alternative
techniques. They are much faster than CLIQUE and they
can discover disjoint clusters. In PROCLUS, the dimensions
relevant to each cluster are selected from the original set of
attributes. ORCLUS is more general and can select relevant attributes from the set of arbitrary directed orthogonal
vectors. PROCLUS fails to identify clusters with large difference in size and requires their dimensionality to be in a
predefined range. ORCLUS may discover clusters that are
hard to interpret.

\section{DOC}

DOC \cite{bib12} is a density-based algorithm that iteratively discovers projected clusters in a data set. DOC discovers one
cluster at a time. At each step, it tries to guess a good
medoid for the next cluster to be discovered. It repeatedly
picks a random point from the database and attempts to
discover the cluster centered at random point. Among all
discovered clusters, the cluster with the highest quality is selected. The process repeated for number of random points.

\section{Frequent Pattern-based Algorithm/MineClus}

The frequent pattern mining\cite{bib7} is one of the interesting
algorithms in data mining. The frequent itemset mining
problem was first proposed in\cite{bib4}. The frequent patterns
are generated from the transactional databases for a given
minsup. Subsequently, an efficient approach was proposed
in FP-growth\cite{bib8} to generate the frequent patterns without
generating the candidate items. In\cite{bib18, bib19}, a projected clustering algorithm was proposed. The algorithm iteratively
produce one cluster at a time by exploiting the frequent pattern mining approach. The top supported frequent patterns
are employed to find the item sets to generate the clusters.
A strength function is defined to arrange the clusters in to
strong and weak clusters. The strong clusters are produced
as an output and the objects in the weak clusters are added
by to the database. The process is iterated till all the objects
are grouped into the clusters. The FP-growth approach has
employed to generate the frequent patterns.
 % Experimental Setup

\chapter{MineClus and Diversity/DiverseRank}
\label{Chapter3}
\lhead{Chapter 3. \emph{MineClus and Diversity/DiverseRank}}

\section{MinClus}
The approach works based on the given a random medoid p $ \in $ S, the approach transform best projected cluster containing p, to the trasnactional databases. 
If the attribute of a record is bounded by p with respect to the width w (here, w=2), an item for that attribute is added to the corresponding itemset.
The appraoch observe that all frequent itemsets (i.e., combinations of dimensions) with respect to minsup = $\alpha * |S|$ are candidate clusters for medoid p.
The problem of finding the best projected cluster for a random medoid p can be transformed to the problem of finding the best itemset in a transformation of S, where goodness is defined by the $\mu$ function (refer Equation \ref{eq:1}).

Instead of discovering it in an non-deterministic way, a systematic data mining algorithm on S is applied. 
The association between the data items is identified by the frequent pattern mining algoirhtm. 
Due to association in the data items, the frequent pattern mining approach has been exploited in projected cluter approachs.
Recently, there is a more efficient algorithm, the FP-growth method \cite{bib8}. 
Here, the appraoch adopt frequent pattern mining for subspace clustering.
However, the objective is to find the frequent itemset with maximum $\mu$ value, rather than finding all frequent subspaces.
Assume that $(I_{best})$ is the itemset with the maximum $\mu$ value found so far and let $dim(I_{best})$ and $sup(I_{best})$ be its dimensionality and support, respectively. 

Let $I_{cond}$ be the current conditional pattern of the FP-growth process. Its support $sup(I_{cond})$ gives an upper bound for the supports of all patterns containing it. 
Moreover, the dimensionality of the itemsets that contain $I_{cond}$ is at most $dim(I_{cond})+l$, where l is the number of items above the items in in the header table of the FP-Growth. 
The $\mu$ function helps to find the strong and weak clusters. 
Among the strong clusters, the one with highest $\mu$ score is prouced as the final cluster and the data in the other clusters are added back to the database for generation of the next clusters. 
The process is repeated till the now data point is left in the database. 
The final data points which are not part of any cluster is treated as outliers.
\begin{equation}\label{eq:1}
\mu(a, b) = a * (1/\beta) * b
\end{equation}
where, a is support of the frequent patter, $\beta \in (0, 1]$ reflects
the importance of the projection, and b is the number of items in the frequent pattern.
%\figure{}

\section{Diversity/DiverseRank}
\subsection{About Concept Hierarchies}
A pattern contains data items. A concept hierarchy is a tree in which the data items are organized in a hierarchical manner. In this tree, all the leaf nodes are the items, the internal nodes represent the categories and the top node represents the root. The root could be a virtual node. Figure1 \iffalse(figure number) \fi is an example of the concept hierarchy.

Let C be a concept hierarchy. A node in C may be an item, category or root. The height of the root node is 0. Let n be a node in C. The height of n, is denoted as h(n), is equal to the number of edges on the path from root to n.

The concept hierarchies can be balanced and unbalanced. In the balanced concept hierarchy having height h has the same number of levels. The items at the given height are said to be at the same level. In an unbalanced concept hierarchy, the height of at least one of the leaf level node is different from the height of other leaf level nodes. The height of unbalanced concept hierarchy is equal to the height of the leaf level node having maximum height. In concept hierarchy C, all the lower-level nodes, except the root, are mapped to the immediate higher level nodes. We consider the concept hierarchies in which a lower level node is mapped to only one higher level node.

\subsection{Diversity}
The diversity of a pattern is based on the category of the items within it. If the items of a pattern are mapped to the same/few categories in a concept hierarchy, we consider that the pattern has low diversity. Relatively, if the items are mapped to multiple categories, we consider that the pattern has more diversity. 

\subsubsection{1}
For a given pattern diversity/diverserank is assigned based on the merging behavior in the corresponding concept hierarchy. If the pattern merges into few higher level categories quickly, it has low diversity/diverserank. Otherwise, if the pattern merges into one or a few high level categories slowly, it has relatively high diversity/diverserank value.

As an example, consider the concept hierarchy in Figure 1 \iffalse(figure number)\fi. For the pattern \{a, b\} the items a and b are mapped to the next level category p. In this case, the merging occurs quickly.For the pattern \{a, d\}, the item a is mapped to category p while item d is mapped to category q . Further, both th categories p and q are mapped to the category w. We say that the pattern \{a, d\} is more diverse than the pattern \{a, b\} as the merging is relatively slow in case of \{a, d\} as compared to \{a, b\}. Consider the pattern \{a, i\} which is relatively more diverse than the pattern \{a, d\} as both items merges at the root. The merging of \{a, i\} occurs slowly as comapred to \{a, d\}

\textbf{Computing the Diversity of Patterns}

We explain the process of calculating diverse rank of the pattern, called DRank, proposed in [\cite{bib14}, \cite{bib15}], the balanced and unbalanced pattern as follows.

We extract the projection of Concept Hierarchy for $Y$(pattern) to compute the diversity.

\textbf{Projection of Concept Hierarchy for $Y (P (Y /C))$}: Let $Y$ be $P$ and $C$ be concept hierarchy. The $P (Y /C)$ is the projection of $C$ for $Y$ which contains the portion of $C$. 
All the nodes and edges exists in the paths of the items of $Y$ to the root, along with the items and the root, are included in $P (Y /C)$. 
The projection $P (Y /C)$ is a tree which represents a concept hierarchy concerning to the pattern $Y$.

Given two patterns of the same length, different merging behavior can be realized, if we observe how the items in the patterns are mapped to higher level nodes. 
That is, one pattern may quickly merge to few higher level items within few levels and the other patterns may merge to few higher level items by crossing more number of levels. 
By capturing the process of merging, we define the notion of diverse rank (drank). 
So, $drank (Y )$ is calculated by capturing how the items are merged from leaf-level to root in $P (Y/C)$. 
It can be observed that a given item set maps from the leaf level to the root level through a merging process by crossing intermediate levels. 
At a given level, several lower level items/categories are merged into the corresponding higher level categories.

Two notions are employed to compute the diversity of pattern: Merging Factor ($MF$), Level Factor ($LF$) and Adjustment factor ($AF$).

We explain about MF after presenting the notion of generalized pattern.

\textbf{Generalized Pattern} $(GP (Y, l, P (Y /C)))$: Let $Y$ be a pattern, $h$ be the height of $P (Y /C)$ and $l$ be an integer.
The $GP (Y, l, P (Y /C))$ indicates the $GP$ of $Y$ at level  $l$ in $P (Y /C)$. 
Assume that the $GP (Y, l + 1, P (Y /C))$ is given. 
The $GP (Y, l, P (Y /C))$ is calculated based on the $GP$ of $Y$ at level $(l + 1)$. 
The $GP (Y, l, P (Y /C))$ is obtained by replacing every item at level $(l + 1)$ in $GP (Y, l + 1, P (Y /C))$ with its corresponding parent at the level $l$ with duplicates removed, if any.

The notion of merging factor at level l is defined as follows.

\textbf{Merging factor} $(MF (Y, l, P (Y /C)))$: Let $Y$ be pattern and $l$ be the level. The merging factor indicates how the items of a
pattern merge from the level $l + 1$ to the level $l (0 ≤ l < h)$.
If there is no change, the $MF(Y,l)$ is $1$. 
If all items merges to one node, the $MF(X,l)$ value equals to $0$.
So, the $MF$ value at the level $l$ is denoted by $MF(Y, l, P (Y /C))$ which is equal
to the ratio of the number of nodes in $(GP (Y, l, P (Y /C) − 1)$ to the number of nodes in $(GP (Y, l + 1, P (Y /C) − 1)$.

\begin{equation}
    MF(Y,l,P(Y/C)) = \frac{|GP(Y,l,P(Y/C)| - 1}{|GP(Y,l+1,P(Y/C)))|-1}
\end{equation}

We now define the notion of level factor to determine the contribution of nodes at the given level.

Level Factor $(LF (l, P (Y /C))$: For a given $P (Y /C)$, $h$ be the height of $P (Y /C) = {0, 1}$. 
Let $l$ be such that $1 ≤ l ≤ (h − 1)$.
The $LF$ value of $P (Y /C)$ at level $l$ indicates the contribution of nodes at level $l$ to $DRank$.
We can assign equal, linear or exponential weights to each level. 
Here, we provide a formula which assigns the weight to the level such that the weight is in proportion the level number.

\begin{equation}
LF (l, P (Y /C))=\frac{2 * (h - l)}{ h * (h - 1)}
\end{equation}

\textbf{Adjustment factor} $AF (Y, l, P (Y /C))$:Let $Y$ be pattern and $l$ be the level
The Adjustment Factor $(AF)$ at level $l$ helps in reducing the drank by measuring the contribution of dummy edges/nodes relative to the original edges/nodes at the level $l$. 
The $AF$ for a pattern $Y$ at a level $l$ should depend on the ratio of number of real edges formed with the children of the 
real nodes in $P (Y /E)$ versus total number of edges formed with the children of real and dummy nodes at $l$ in $P (Y /E)$. 
The value of $AF$ at a given height should lie between $0$ and $1$. 
If the number of real edges is equals to zero, $AF$ is $0$.
If the pattern at the given level does not contain dummy nodes/edges, the value of $AF$ becomes $1$.
Note that the $AF$ value is not defined at the leaf level nodes as children do not exist. 
The $AF$ for $Y$ at level $l$ is denoted as $AF (Y, l, P (Y /E))$ and is calculated by the following formula.
The value for $AF$ is less than $1$ when the pattern is unbalanced otherwise it returns $1$.

\begin{equation}
    AF (Y, l, P (Y /C)) = \frac{\# of Real Edges of U P (Y, l, P (Y /E)) }{ \# of T otal Edges of U P (Y, l, P (Y /E))}
\end{equation}

where numerator is the number of edges formed with the children of the real nodes and denominator is the number of edges formed with the children of both real and dummy nodes at the level $l$ in $P (Y /E)$.

The approach to compute drank of $Y$ is as follows. 
We convert the concept hierarchy to extended concept hierarchy called, "extended concept hierarchy" by adding dummy nodes and edges. 
Next, we adjust the drank in accordance with the number of dummy nodes and edges using the notion called adjustment factor. 
So, the drank of $Y$ is relative to the drank of the same pattern computed by considering all of its items are at the leaf level of the extended concept hierarchy. 
The dummy nodes and edges are added when the pattern is unbalanced.

\textbf{Diverse rank of a frequent pattern} $Y (drank(Y))$: Let $Y$ be the pattern and $C$ be the unbalanced concept hierarchy
of height $h$. 
The drank of $Y$ , denoted by $drank(Y )$, is given by the following equation:

\begin{equation}
    drank(Y, C) = [M F (Y, l, P (Y /E)) ∗ AF (Y, l, P (Y /E))] * LF (l, P (Y /E))     
\end{equation}

where, $h$ is the height of the $P (P/E)$, $E$ is the extended unbalanced concept hierarchy, $MF (Y, l, P (Y /E))$ is the $MF$ 
of $Y$ at level $l$, $LF (l, P (Y /E))$ is the $LF$ at level $l$ and $AF (Y, l, P (Y /E))$ is the $AF$ of $Y$ at level $l$.

\subsubsection{2}
For a given pattern diversity/diverserank is assigned based on difference in number of edges in the sub tree of corresponding concept hierarchy which only contain those leaves(items) which are present in the pattern (e') and minimum number of edges needed to form a tree of height same as the corresponding concept hierarchy  number of leaf nodes is equal to number of distinct items in the pattern(e). 
To normalize it we divide it with the difference of number of edges if all the pattern merges at root(E) and  minimum number of edges needed to form a tree of height same as the     corresponding concept hierarchy  number of leaf nodes is equal to number of distinct items in the pattern(e)

\begin{equation}\label{eq:2}
    dr(Y,C)=\frac{e'-e}{E-e}
\end{equation}

As an example, consider the concept hierarchy in Figure 1 \iffalse(figure number)\fi. \\ 
For the pattern \{a,b\}, E=6, e'=4 and e=4. So diversty=0.0 \\
For the pattern \{a,d\}, E=6, e'=5 and e=4. So diversty=0.5 \\
For the pattern \{a,i\}, E=6, e'=6 and e=4. So diversty=1.0
 % Experiment 1

%\chapter{Proposed Approach}
\label{Chapter4}
\lhead{Chapter 4. \emph{Proposed Approach}}

Our approach uses Diversity to refine each cluster as it is formed. The MineClus algorithm generates clusters iteratively using a \emph{Frequent Pattern} with high support.
Right after a cluster is formed, we try to validate the membership of each element in the newly formed cluster. Any element that is deemed to be different from the rest of the cluster is put back into the dataset, and the MineClus algorithm will consider it while forming newer clusters in the future. Hence, an element is added to the cluster which is best suited for it. If a suitable cluster cannot be found for the element, it is listed as a singleton cluster.

The rationale and key idea behind using diversity is that all elements of the cluster will have similar diversity values with respect to the Concept Hierarchy. The concept hierarchy captures domain knowledge of the dataset and represents it in a tree-like form.
Diversity is able to identify how different a data point
is from other points in the cluster. If the data point
is too different (exceeding a threshold), it clearly does not belong to the current cluster and is labelled as a non-member.
All non-members are put back into the list of data points that are yet to be clustered.

For the purposes of refining, let us consider a cluster \(C\) containing \(n\) data points \(\{o_1, o_2, \dots, o_n \}\). We use diversity to determine the membership of \(o_i\). The array containing the diversity values for each of the cluster elements is denoted as \(\{d_1, d_2, \dots, d_n\}\). Two approaches for this are proposed.

\section{Diverse Square-Root Distance}

We define a measure called \emph{Diverse Square Root Distance (dsrd(k))} as:

\begin{equation}
dsrd(k) = \sqrt{\sum_{j=1}^{n} {(d_k - d_j)}^2}
\end{equation}

\(dsrd(k)\) represents the distance of the \(k^{th}\) element from the rest of the cluster.

\begin{algorithm}[H]
 \KwData{\(C, D, \delta\)}
$C$ is the current cluster\;
$D$ contains elements to be clustered\;
$\delta$ is a user-defined threshold\;
 \KwResult{Refined $C$}
 %\While{not at end of this document}{
\For{$i \leftarrow 1$ to $n$}{
  %read current\;
  %\eIf{understand}{
   %go to next section\;
   %current section becomes this one\;
   %}{
   %go back to the beginning of current section\;
  %}
\eIf{$dsrd(i) > \delta$} {
	Remove $i^{th}$ element from $C$\;
}{
	Keep $i^{th}$ element in $C$\;
}
 }
return $C$\;
 \caption{How to write algorithms}
\end{algorithm}
 % Experiment 2

\chapter{Experiments}
\label{Chapter5}
\lhead{Chapter 5. \emph{Experiments}}

\section{Work Done}
\begin{itemize}
	\item We studied relevant literature, as mentioned in the references
	\item We understood and implemented the Apriori Algorithm and FP-Growth algorithms in Python
	\item We understood and implemented both the implementions of Diversity in Python
	\item We implemented the entire MinClus clustering suite with support for merging
	\item We proposed and implemented six Outlier detection/Cluster refining methods
	\item We conducted extensive testing of our approach and that of the original MinClus approach, results of which are presented below.
\end{itemize}

\section{Evaluation Metric}
We view Clustering as a series of decisions - for each pair of items in the dataset, whether to put them in the same cluster or in a different cluster.

A true positive (TP) decision assigns two
similar documents to the same cluster, a true negative (TN)
decision assigns two dissimilar documents to different clus-
ters. There are two types of errors we can commit. A (FP)
decision assigns two dissimilar documents to the same clus-
ter. A (FN) decision assigns two similar documents to dif-
ferent clusters.

\section{Results with Diversity1(original approach)}
\subsection{Without merging}

\begin{table}[H]
\caption{Accuracy} 
\begin{center}
		\begin{tabular}{ | l | l | l | l |}
				\hline

				\textbf{Dataset} & \textbf{MinClus} & \textbf{DSRD} & \textbf{Binning} \\ \hline
                \textbf{Iris} & 82.20 & 85.58 & 85.58 \\ \hline
                \textbf{Seed} & 79.11 & 78.96 & 78.96 \\ \hline
                \textbf{Zoo} & 88.33 & 88.55 & 81.18  \\ \hline
                \textbf{Water} & 60.23 & 62.28 & 60.24  \\ \hline
		\end{tabular}
\end{center}
\label{table:acc1}
\end{table}



\begin{table}[H]
\caption{Precision} 
\begin{center}
		\begin{tabular}{ | l | l | l | l |}
				\hline

				\textbf{Dataset} & \textbf{MinClus} & \textbf{DSRD} & \textbf{Binning} \\ \hline

				\textbf{Iris} & 97.50 & 98.28 & 98.28 \\ \hline
				\textbf{Seed} & 90.62 & 89.03 & 89.03 \\ \hline
				\textbf{Zoo} & 94.02 & 91.28 & 88.94  \\ \hline
				\textbf{Water} & 36.32 & 39.07 & 36.60 \\ \hline
		\end{tabular}
\end{center}
\label{table:pre1}
\end{table}


\begin{table}[H]
\caption{Recall} 
\begin{center}
		\begin{tabular}{ | l | l | l | l |}
				\hline

				\textbf{Dataset} & \textbf{MinClus} & \textbf{DSRD} & \textbf{Binning} \\ \hline

				\textbf{Iris} &  47.84 & 57.76 & 57.76 \\ \hline
				\textbf{Seed} & 41.65 & 42.05 & 42.05 \\ \hline
				\textbf{Zoo} & 55.03 & 57.96 & 24.89  \\ \hline
				\textbf{Water} & 19.61 & 15.79 & 20.09 \\ \hline
		\end{tabular}
\end{center}
\label{table:rec1}
\end{table}

\subsection{With merging}

\begin{table}[H]
\caption{Accuracy} 
\begin{center}
		\begin{tabular}{ | l | l | l | l |}
				\hline

				\textbf{Dataset} & \textbf{MinClus} & \textbf{DSRD} & \textbf{Binning} \\ \hline
                \textbf{Iris} & 82.78 & 91.80 & 91.80 \\ \hline
                \textbf{Seed} & 82.51 & 83.21 & 83.21 \\ \hline
                \textbf{Zoo} & 91.02 & 94.65 & 90.47  \\ \hline
                \textbf{Water} & 60.29 & 59.80 & 60.50  \\ \hline
		\end{tabular}
\end{center}
\label{table:acc2}
\end{table}



\begin{table}[H]
\caption{Precision} 
\begin{center}
		\begin{tabular}{ | l | l | l | l |}
				\hline

				\textbf{Dataset} & \textbf{MinClus} & \textbf{DSRD} & \textbf{Binning} \\ \hline

				\textbf{Iris} & 75.89 & 87.38 & 87.38 \\ \hline
				\textbf{Seed} & 78.72 & 79.86 & 79.86 \\ \hline
				\textbf{Zoo}  & 90.95 & 90.02 & 81.06 \\ \hline
				\textbf{Water}& 36.94 & 36.70 & 39.38 \\ \hline
		\end{tabular}
\end{center}
\label{table:pre2}
\end{table}

\begin{table}[H]
\caption{Recall} 
\begin{center}
		\begin{tabular}{ | l | l | l | l |}
				\hline

				\textbf{Dataset} & \textbf{MinClus} & \textbf{DSRD} & \textbf{Binning} \\ \hline

				\textbf{Iris}  & 70.85 & 88.11 & 88.11 \\ \hline
				\textbf{Seed}  & 65.13 & 66.35 & 66.35  \\ \hline
				\textbf{Zoo}   & 69.61 & 87.45 & 78.82  \\ \hline
				\textbf{Water} & 20.63 & 22.06 & 25.90 \\ \hline
		\end{tabular}
\end{center}
\label{table:rec2}
\end{table}


\section{Results with Diversity2(improved approach)}
\subsection{Without merging}

\begin{table}[H]
\caption{Accuracy} 
\begin{center}
		\begin{tabular}{ | l | l | l | l |}
				\hline

				\textbf{Dataset} & \textbf{MinClus} & \textbf{DSRD} & \textbf{Binning} \\ \hline
                \textbf{Iris} & 82.20 & 82.43 & 77.54 \\ \hline
                \textbf{Seed} & 79.11 & 79.96 & 75.06 \\ \hline
                \textbf{Zoo} & 88.33 & 88.33 & 85.65  \\ \hline
                \textbf{Water} & 60.23 & 63.35 & 60.27  \\ \hline
		\end{tabular}
\end{center}
\label{table:acc1}
\end{table}



\begin{table}[H]
\caption{Precision} 
\begin{center}
		\begin{tabular}{ | l | l | l | l |}
				\hline

				\textbf{Dataset} & \textbf{MinClus} & \textbf{DSRD} & \textbf{Binning} \\ \hline

                \textbf{Iris} & 97.50 & 98.74 & 98.34 \\ \hline
                \textbf{Seed} & 90.62 & 95.76 & 91.00 \\ \hline
                \textbf{Zoo} & 94.02 & 94.02 & 96.79  \\ \hline
                \textbf{Water} & 36.32 & 39.96 & 36.37 \\ \hline
		\end{tabular}
\end{center}
\label{table:pre1}
\end{table}


\begin{table}[H]
\caption{Recall} 
\begin{center}
		\begin{tabular}{ | l | l | l | l |}
				\hline

				\textbf{Dataset} & \textbf{MinClus} & \textbf{DSRD} & \textbf{Binning} \\ \hline

                \textbf{Iris} &  47.84 & 47.89 & 33.17 \\ \hline
                \textbf{Seed} & 41.65 & 41.74 & 27.93 \\ \hline
                \textbf{Zoo} & 55.03 & 55.03 & 41.75  \\ \hline
                \textbf{Water} & 19.61 & 11.43 & 19.53 \\ \hline
		\end{tabular}
\end{center}
\label{table:rec1}
\end{table}

\subsection{With merging}

\begin{table}[H]
\caption{Accuracy} 
\begin{center}
		\begin{tabular}{ | l | l | l | l |}
				\hline

				\textbf{Dataset} & \textbf{MinClus} & \textbf{DSRD} & \textbf{Binning} \\ \hline
                \textbf{Iris} & 82.78 & 90.94 & 93.24 \\ \hline
                \textbf{Seed} & 82.51 & 85.00 & 83.41 \\ \hline
                \textbf{Zoo} & 91.02 & 91.96 & 95.78  \\ \hline
                \textbf{Water} & 60.29 & 62.05 & 60.18  \\ \hline
		\end{tabular}
\end{center}
\label{table:acc2}
\end{table}



\begin{table}[H]
\caption{Precision} 
\begin{center}
		\begin{tabular}{ | l | l | l | l |}
				\hline

				\textbf{Dataset} & \textbf{MinClus} & \textbf{DSRD} & \textbf{Binning} \\ \hline

                \textbf{Iris} & 75.89 & 86.21 & 89.82 \\ \hline
                \textbf{Seed} & 78.72 & 77.50 & 80.45 \\ \hline
                \textbf{Zoo}  & 90.95 & 86.58 & 93.10 \\ \hline
                \textbf{Water}& 36.94 & 38.86 & 36.71 \\ \hline
		\end{tabular}
\end{center}
\label{table:pre2}
\end{table}

\begin{table}[H]
\caption{Recall} 
\begin{center}
		\begin{tabular}{ | l | l | l | l |}
				\hline

				\textbf{Dataset} & \textbf{MinClus} & \textbf{DSRD} & \textbf{Binning} \\ \hline

                \textbf{Iris}  & 70.85 & 86.69 & 89.92 \\ \hline
                \textbf{Seed}  & 65.13 & 77.51 & 66.38 \\ \hline
                \textbf{Zoo}   & 69.61 & 78.82 & 89.08 \\ \hline
                \textbf{Water} & 20.63 & 16.57 & 20.59 \\ \hline
		\end{tabular}
\end{center}
\label{table:rec2}
\end{table}

 % Results and Discussion

%\input{./Chapters/Chapter7} % Conclusion

%% ----------------------------------------------------------------
% Now begin the Appendices, including them as separate files

\addtocontents{toc}{\vspace{2em}} % Add a gap in the Contents, for aesthetics

\appendix % Cue to tell LaTeX that the following 'chapters' are Appendices

\input{./Appendices/AppendixA}	% Appendix Title

%\input{./Appendices/AppendixB} % Appendix Title

%\input{./Appendices/AppendixC} % Appendix Title

\addtocontents{toc}{\vspace{2em}}  % Add a gap in the Contents, for aesthetics
\backmatter

%% ----------------------------------------------------------------
\label{Bibliography}
\lhead{\emph{Bibliography}}  % Change the left side page header to "Bibliography"
\bibliographystyle{unsrtnat}  % Use the "unsrtnat" BibTeX style for formatting the Bibliography
%\bibliography{Bibliography}  % The references (bibliography) information are stored in the file named "Bibliography.bib"
\begin{thebibliography}{99}
	\bibitem{bib1}
		C. C. Aggarwal, J. L. Wolf, P. S. Yu, C. Procopiuc, and J. S. Park. Fast algorithms for projected clustering. \emph{SIGMOD} Rec., 28(2):61–72, June 1999.
	\bibitem{bib2}
		C. C. Aggarwal and P. S. Yu. Finding generalized projected clusters in high dimensional spaces. \emph{SIGMOD} Rec., 29(2):70–81, May 2000.
	\bibitem{bib3}
		R. Agrawal, J. Gehrke, D. Gunopulos, and P. Raghavan. Automatic subspace clustering of high dimensional data for data mining applications.  SIGMOD Rec., 27(2):94–105, June 1998.  
	\bibitem{bib4}
		R. Agrawal and R. Srikant. Fast algorithms for mining association rules in large databases. \emph{In Proceedings of the 20th International Conference on Very Large Data Bases}, VLDB ’94, pages 487–499, San Francisco, CA, USA, 1994. Morgan Kaufmann Publishers Inc.
	\bibitem{bib5}
		K. S. Beyer, J. Goldstein, R. Ramakrishnan, and U. Shaft. When is “nearest neighbor” meaningful? \emph{In Proceedings of the 7th International Conference on Database Theory}, ICDT ’99, pages 217–235, London, UK, UK, 1999. Springer-Verlag.
	\bibitem{bib6}
		C. Faloutsos and K.-I. Lin. Fastmap: A fast algorithm for indexing, data-mining and visualization of traditional and multimedia datasets. \emph{SIGMOD} Rec., 24(2):163–174, May 1995.
	\bibitem{bib7}
		J. Han, H. Cheng, D. Xin, and X. Yan. Frequent
		pattern mining: Current status and future directions.
		\emph{Data Min. Knowl. Discov.}, 15(1):55–86, 2007.
	\bibitem{bib8}
		J. Han, J. Pei, and Y. Yin. Mining frequent patterns
		without candidate generation. \emph{SIGMOD} Rec.,
		29(2):1–12, May 2000.
	\bibitem{bib9}
		A. Hinneburg, C. C. Aggarwal, and D. A. Keim. What
		is the nearest neighbor in high dimensional spaces? \emph{In
			Proceedings of the 26th International Conference on
		Very Large Data Bases}, VLDB ’00, pages 506–515,
		San Francisco, CA, USA, 2000. Morgan Kaufmann
		Publishers Inc.
	\bibitem{bib10}
		A. K. Jain, M. N. Murty, and P. J. Flynn. Data
		clustering: A review. \emph{ACM Comput. Surv.},
		31(3):264–323, 1999.
	\bibitem{bib11}
		G. Moise, A. Zimek, P. Kruger, H.-P. Kriegel,
		and J. Sander. Subspace and projected clustering:
		Experimental evaluation and analysis. \emph{Knowl. Inf.
		Syst.}, 21(3):299–326, 2009.
	\bibitem{bib12}
		C. M. Procopiuc, M. Jones, P. K. Agarwal, and T. M.
		Murali. A monte carlo algorithm for fast projective
		clustering. In \emph{Proceedings of the 2002 ACM SIGMOD
		International Conference on Management of Data},
		SIGMOD ’02, pages 418–427, New York, NY, USA,
		2002. ACM.
	\bibitem{bib13}
		W. M. Rand. Objective criteria for the evaluation of
		clustering methods. \emph{Journal of the American
		Statistical Association}, 66(336):846–850, 1971.
	\bibitem{bib14}
		S. Srivastava, R. U. Kiran, and P. K. Reddy.
		Discovering diverse-frequent patterns in transactional
		databases. In \emph{Proceedings of the 17th International
		Conference on Management of Data}, COMAD ’11,
		pages 14:1–14:10. Computer Society of India, 2011.
	\bibitem{bib15}
		M. K. Swamy, P. K. Reddy, and S. Srivastava.
		Extracting diverse patterns with unbalanced concept
		hierarchy. In \emph{Advances in Knowledge Discovery and
		Data Mining - 18th Pacific-Asia Conference}, PAKDD
		2014, Tainan, Taiwan, May 13-16, 2014. Proceedings,
		Part I, pages 15–27. Springer-Verlag, 2014.
	\bibitem{bib16}
		UCI. Machine Learning Repository,
		\url{https://archive.ics.uci.edu/ml/index.html}, 2014
		(accessed August, 2014).
	\bibitem{bib17}
		R. Xu and D. Wunsch, II. Survey of clustering
		algorithms. \emph{Trans. Neur. Netw.}, 16(3):645–678, 2005.
	\bibitem{bib18}
		M. L. Yiu and N. Mamoulis. Frequent-pattern based
		iterative projected clustering. In \emph{Proceedings of the
			Third IEEE International Conference on Data
		Mining}, ICDM ’03, pages 689–692, Washington, DC,
		USA, 2003. IEEE Computer Society.
	\bibitem{bib19}
		M. L. Yiu and N. Mamoulis. Iterative projected
		clustering by subspace mining. \emph{IEEE Trans. on
		Knowl. and Data Eng.}, 17(2):176–189, 2005.
\end{thebibliography}

\end{document}  % The End
%% ----------------------------------------------------------------
